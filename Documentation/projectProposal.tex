\documentclass[journal]{IEEEtran}
\usepackage{url}

% *** GRAPHICS RELATED PACKAGES ***
%
\ifCLASSINFOpdf
  % \usepackage[pdftex]{graphicx}
  % declare the path(s) where your graphic files are
  % \graphicspath{{../pdf/}{../jpeg/}}
  % and their extensions so you won't have to specify these with
  % every instance of \includegraphics
  % \DeclareGraphicsExtensions{.pdf,.jpeg,.png}
\else
  % or other class option (dvipsone, dvipdf, if not using dvips). graphicx
  % will default to the driver specified in the system graphics.cfg if no
  % driver is specified.
  % \usepackage[dvips]{graphicx}
  % declare the path(s) where your graphic files are
  % \graphicspath{{../eps/}}
  % and their extensions so you won't have to specify these with
  % every instance of \includegraphics
  % \DeclareGraphicsExtensions{.eps}
\fi



% *** Do not adjust lengths that control margins, column widths, etc. ***
% *** Do not use packages that alter fonts (such as pslatex).         ***
% There should be no need to do such things with IEEEtran.cls V1.6 and later.
% (Unless specifically asked to do so by the journal or conference you plan
% to submit to, of course. )


% correct bad hyphenation here
\hyphenation{op-tical net-works semi-conduc-tor}


\begin{document}

% paper title
% can use linebreaks \\ within to get better formatting as desired
% Do not put math or special symbols in the title.

\title{Project Proposal for Home Automation System}

\author{Vivek~Pujara,~\IEEEmembership{CS~807~Student,~University~of~Regina,}
\\~Mikhail~Shchukin,~\IEEEmembership{CS~807~Student,~University~of~Regina,}
\\~Gideon~Eromosele,~\IEEEmembership{CS~807~Student,~University~of~Regina,}
~and~Oluwatobi~Adegbola,~\IEEEmembership{CS~807~Student,~University~of~Regina,}
}


% The paper headers
% This should either be the name of the journal or the
% first four(ish) words of the paper
\markboth{CS 807 Project Proposal, Mar~6th~2019}%
{Shell \MakeLowercase{\textit{et al.}}: Design, Template, Computer Science }
% The only time the second header will appear is for the odd numbered pages
% after the title page when using the twoside option.


% make the title area
\maketitle

% As a general rule, do not put math, special symbols or citations
% in the abstract or keywords.
\begin{abstract}
This research project proposal provides a comprehensive overview on the
suggested modifications to be built upon an existing hardware design,
the motivation behind improving the base design, the materials
to be gathered for such improvement, the tentative scheduling of
the workload to be followed and the distribution of research and
hardware design tasks in a given team setting.
\end{abstract}

% Note that keywords are not normally used for peerreview papers.
\begin{IEEEkeywords}
design, home automation, hardware, CS 807
\end{IEEEkeywords}



% For peer review papers, you can put extra information on the cover
% page as needed:
% \ifCLASSOPTIONpeerreview
% \begin{center} \bfseries EDICS Category: 3-BBND \end{center}
% \fi
%
% For peerreview papers, this IEEEtran command inserts a page break and
% creates the second title. It will be ignored for other modes.
\IEEEpeerreviewmaketitle



\section{Motivation}

% Some journals put the first two words in caps:
% \IEEEPARstart{T}{his demo} file is ....
%
% Here we have the typical use of a "T" for an initial drop letter
% and "HIS" in caps to complete the first word.
\IEEEPARstart{T}{he} motivation behind choosing home automation as a research project topic is based on the increasing community interest in the Internet of Things (IoT) as a concept allowing to extend the automation of any system, including a typical residential building, to a considerably new level of integration.

The idea revolves around having a network of various actuators and sensors, in various settings, where the user has a remote control capability for basically any interactive component of such hardware network. Particularly, an automated home can be defined as a hardware-augmented dwelling, where various sensors are the inputs keeping track of the environment and user state and the automated parts (doors, lighting, security alarm, etc.) are the actuators, the outputs of the system, and these components are connected together via some processing unit, usually a microcontroller.

For example, such system gives the resident of such hardware-augmented dwelling an ability for a TV or some living room lights to turn on based on sensing the presence of the owner in the house. Home automation enables the owner with the extended control over the property, where some functions can be accessed remotely, and the system might be able to notify the owner of its state (intrusion alert, regular status notifications, troubleshooting and more). The proposed research project strives to pick an existing hardware project related to the Internet of things and home automation and suggest meaningful modifications and improvements to the base design.
% You must have at least 2 lines in the paragraph with the drop letter
% (should never be an issue)

%%%%%%%%%%%%%%%%%%%%%%%%%%%%%%%%%%%%%%%%%%%%%%%%%%%%%%%%%%%%%%%%%%%%%%%%%%%%%%%%%%%%%%%%%%%%%%%%%%%%%%%%%%%
\section{Base Project}
The concept of Smart Home was introduced in 1998 and was majorly developed in the early 2000’s \textsuperscript{\cite{IEEEhowto:Hendricks}}. Since then many attempts have been made to design a system that could be used in almost any type of a house. The project taken as a base is a similar attempt. It was found on Arduino Project Hub named Home Automation Using Arduino and Bluetooth Control \textsuperscript{\cite{IEEEhowto:Kumar}}. It is an Arduino based system comprising of several sensors and actuators along with a Bluetooth module for control via mobile device. The functionalities include controlling lights, door, TV and temperature. Sensors namely Infrared (IR), motion sensor (PIR) and ultrasound are used for detecting human presence. Temperature sensor has also been used to monitor temperature inside the house. Servo motor and LED’s are used to simulate the effects of controlling lights, TV and door.
Purpose behind selecting this project as a base is that it has all the basic functionalities a home automation system should have. Yet there is huge scope of improvement. There is lack of security, redundancy in usage of different types of sensors and the sensors available in the system can be used as input for several other control systems.


%%%%%%%%%%%%%%%%%%%%%%%%%%%%%%%%%%%%%%%%%%%%%%%%%%%%%%%%%%%%%%%%%%%%%%%%%%%%%%%%%%%%%%%%%%%%%%%%%%%%%%%%%%%
\section{Project Proposal}

\subsection{Suggested Modifications}
Security is a major concern for every house owner, thus a home automation system should have better security. Following security additions are to be done:
\begin{enumerate}
    \item Entry into the house should only be allowed after verifying a person’s identity, planned to be done using RFID tags and reader.
    \item There should also be an alarm system which could be turned off by entering a code.
    \item Perimeter of the house should be monitored for possible intrusions.
\end{enumerate}

Along with security, comfort is crucial. Following functionalities can be added to create increase the user’s comfort:
\begin{enumerate}
    \item Controlling light intensity
    \item Automatic operation of perimeter lights.
    \item Monitoring temperature and controlling Air Conditioning systems
    \item Garage door control
    \item Providing better access and control of the system via Mobile devices using Bluetooth or WiFi.
\end{enumerate}
The system should also have all the functionalities offered in the base project.

\subsection{Materials Required}
The list below shows components required for the implementation of the Home Automation System:
\begin{itemize}
	\item{Sensors:}
	\begin{itemize}
		\item{RFID reader and Tag}
		\item{Keypad}
		\item{Potentiometer}
		\item{Infrared sensor}
		\item{Photoresistor}
		\item{Motion Sensor}
		\item{Temperature Sensor}
		\item{Ultrasonic Sensor}
		\item{Bluetooth module or Wi-Fi (NodeMCU)}
	\end{itemize}
	\item{Actuators:}
	\begin{itemize}
		\item{LEDs}
		\item{Buzzer}
		\item{LCD Screen}
		\item{Servo Motor}
		\item{DC Motor}
	\end{itemize}
	\item{Arduino Mega 2560 (Microcontroller board)}
	\item{Prototyping Breadboards}
	\item{Jumper Wires}
	\item{Various Resistors}
	\item{Various Capacitors}
\end{itemize}

\subsection{Scheduling}
The list below outlines the tentative milestones to be reached during project development:
\begin{itemize}
	\item{\textbf{Milestone \#1:}}
		\begin{itemize}
		\item{Gathering the required materials}
		\item{Due March 10th}
		\end{itemize}
	\item{\textbf{Milestone \#2:}}
		\begin{itemize}
		\item{Basic hardware assembly is done}
		\item{No meaningful code yet}
		\item{Due March 17th}
		\end{itemize}
	\item{\textbf{Milestone \#3:}}
		\begin{itemize}
		\item{Sensors calibrated}
		\item{Various sensors/actuators are functional on their own}
		\item{Wireless networking is in active development}
		\item{Due March 24th}
		\end{itemize}
	\item{\textbf{Milestone \#4:}}
		\begin{itemize}
		\item{ A completely functional demo prototype is ready}
		\item{Due March 31st}
		\end{itemize}
	\item{\textbf{GOAL Milestone:}  }
		\begin{itemize}
		\item{Home Automated System final prototype works with all planned functionality and no failures}
		\item{The documentation of the implemented hardware design is finalized}
		\item{Due April 3rd}
		\end{itemize}
\end{itemize}

\subsection{Workload Distribution}
The project research, hardware design development and other tasks/roles are divided among the research team as follows:
\begin{itemize}
\item{Vivek Pujara}
	\begin{itemize}
	\item{Lead Programmer}
	\item{Project Management Lead}
	\end{itemize}
\item{Mikhail Shchukin}
	\begin{itemize}
	\item{Documentation Lead Specialist}
	\item{Prototype Testing}
	\end{itemize}
\item{Gideon Eromosele}
	\begin{itemize}
	\item{Primary Hardware Designer}
	\item{Auxillary Programmer}
	\end{itemize}
\item{Oluwatobi Adegbola}
	\begin{itemize}
	\item{Auxillary Hardware Designer}
	\item{User Requirement Ellicitation}
	\end{itemize}
\end{itemize}


\section{Summary}
Home automation provides homeowners the opportunity to monitor and control the activities in their home in near real-time \textsuperscript{\cite{IEEEhowto:Pavithra}}. Homeowners can get intruder notifications which helps keep the home safe, and control appliances which help to manage energy among other things. This was why the group decided to embark on a home automation-based project to help reduce the security concern of a homeowner. The system is going to be based on the Arduino Microcontroller and is an upgrade to the project by Shubham Kumar \textsuperscript{\cite{IEEEhowto:Kumar}}. As a differentiating factor, we will add controlling light intensity, automatic operation of perimeter lights, temperature monitoring and controlling air conditioning systems, garage door control, and better access and control of the system via Mobile devices using Bluetooth or Wi-Fi. All these to help the homeowner with comfort and less security concern. The project will go through diverse stages including; gathering hardware and designing a schematic of the system, door security and alarm system, light control (both in-house and perimeter), garage door access and Temperature control, perimeter security and remote access via mobile device (Only if time permits).


% Can use something like this to put references on a page
% by themselves when using endfloat and the captionsoff option.
\ifCLASSOPTIONcaptionsoff
  \newpage
\fi

% references section

% can use a bibliography generated by BibTeX as a .bbl file
% BibTeX documentation can be easily obtained at:
% http://www.ctan.org/tex-archive/biblio/bibtex/contrib/doc/
% The IEEEtran BibTeX style support page is at:
% http://www.michaelshell.org/tex/ieeetran/bibtex/
%\bibliographystyle{IEEEtran}
% argument is your BibTeX string definitions and bibliography database(s)
%\bibliography{IEEEabrv,../bib/paper}
%
% <OR> manually copy in the resultant .bbl file
% set second argument of \begin to the number of references
% (used to reserve space for the reference number labels box)
\begin{flushleft}
\begin{thebibliography}{1}

\bibitem{IEEEhowto:Hendricks}
Hendricks, D. (2014). The History of Smart Homes. Retrieved from
\url{https://www.iotevolutionworld.com/m2m/articles/376816-history-smart-homes.htm}
\bibitem{IEEEhowto:Kumar}
Kumar, S. (2019). Home Automation Using Arduino and Bluetooth Control. Retrieved from \url{https://create.arduino.cc/projecthub/Shubhamkumar97/home-automation-using-arduino-and-bluetooth-control-404e9c?ref=user\&ref\_id=598721\&offset=0}
\bibitem{IEEEhowto:Pavithra}
Pavithra, D., \& Balakrishnan, R. (2015). IoT based monitoring and control system for home automation. 2015 Global Conference on Communication Technologies (GCCT), 169-173. doi: 10.1109/gcct.2015.7342646

\end{thebibliography}
\end{flushleft}

% that's all folks
\end{document}
